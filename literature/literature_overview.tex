\documentclass{article}
\usepackage[utf8]{inputenc}
\usepackage{xcolor}
\usepackage{graphicx}
\usepackage{amsmath}
\usepackage{soul}
\usepackage[left=2cm, right=2.5cm, top=3cm, bottom =3cm]{geometry}
\usepackage{hyperref}
\usepackage{natbib}
\usepackage[onehalfspacing]{setspace}
\usepackage{listings}

%\usepackage[english]{babel}
\renewcommand{\figurename}{\textbf{Fig.}}
\renewcommand{\hat}{\widehat}
\usepackage[bf]{caption}
%\usepackage{graphicx} %remove demo for real images! 
\definecolor{mariam:p}{RGB}{0,130,90}
\definecolor{sonja}{cmyk}{0.9,0,0.3,0}
\usepackage{amssymb}
\newcommand{\ar}{$\Rightarrow$ \ }
\newcommand{\dif}{$\Delta$ \ }
\newcommand{\ia}{\item[\ar]}
\newcommand{\ph}{\hat{\bar{\pi}}}
\newcommand{\smt}{\sum^T_{t=1}}
\usepackage{scalerel,stackengine}
\stackMath
\newcommand\reallywidehat[1]{%
\savestack{\tmpbox}{\stretchto{%
  \scaleto{%
    \scalerel*[\widthof{\ensuremath{#1}}]{\kern-.6pt\bigwedge\kern-.6pt}%
    {\rule[-\textheight/2]{1ex}{\textheight}}%WIDTH-LIMITED BIG WEDGE
  }{\textheight}% 
}{0.5ex}}%
\stackon[1pt]{#1}{\tmpbox}%
}

%\usepackage{biblatex}
%\addbibresource{bib.bib}

\title{Literature Overview: Green Finance and Ethics}
\author{Sonja Dobkowitz}
\date{\today}

\begin{document}
\maketitle

\section{Green Finance }
\begin{itemize}
	\item \textbf{``Wie die Wende zu einem nachhaltigen Finanzsystem gelingen kann''} by Claudia Kemfert in  \url{https://makronom.de/green-new-deal-wie-die-wende-zu-einem-nachhaltigen-finanzsystem-gelingen-kann-33471}, Makronom: 
	\begin{itemize}
		\item global perspective
		\item european dependence from gas/oil is profitable to Russia and USA $\rightarrow$ interest in increase in fossil fuel demand
		\item new investments in fossil fuels will increase future dependency and counteracts climate protection
		\item climate protection and economic success are positively related
		\item climate conference Paris, Intergovernmental Panel on Climate Change: EU target: greenhouse gas neutrality/ carbon neutrality until 2050. 
		\item particulate emission = Feinstaubemission
		\item Green finance is key, Green New Deal 
		\item demands that \textbf{all} investment has to be made in green finance; regulatory adjustments of financial system
		\item planned policies \textit{100-day Programme of teh commission} (part of the Green-Deal Architecture)
		\item globally, more and more investment flow into renewable energy sources \textit{private or public?}
		\textit{seems not to focus on climate aspect but rather economic dependency}
		\item commission's action plan of finance of sustainable growth 
	\end{itemize}
\ar distributional aspects within EU! 
\item \textbf{``Ethical investing has reached a tipping point''} in Financial Times, June 18, 2019 source: \url{https://www.ft.com/content/7d64d1d8-91a6-11e9-b7ea-60e35ef678d2}
\begin{itemize} 
	\item \begin{quote}
		
		When the ESG (environmental, social and governance) movement first emerged in the previous century, it was primarily driven by a tiny minority of investors who wanted to promote positive social and environmental change. Swedish pension funds were a case in point.
		
		These proactive``do-gooders'' are still influential, particularly since wealthy western millennials are increasingly embracing the idea of impact investing. But what is really driving the sector’s growth now is a larger group of executives and financiers who want to avoid harm — whether to their own reputations, or the wider world. ESG, in other words, is not longer just a campaigning cause; it is also a risk management tool.

	\end{quote}
 \item \begin{quote}
\textit{ 	However, a third factor is that political and regulatory risks around the issues that ESG tackles are growing. Three dozen central banks recently declared that they will consider environmental factors when regulating the banks. European regulators are starting to impose tough environmental rules, not just on companies but their funders too.}
 	
 \end{quote}
\item \begin{quote}
\textit{ But it could also, at worst, shrivel the value of some corporate assets and investor portfolios if they are linked to, say, carbon emissions.}
\end{quote}
\item \begin{quote}
	\textit{		
		This shift towards risk management dismays some ESG purists, since it has enabled cynics to complain that the market is mere virtue-signalling. But history shows that revolutions occur when the majority of society feels that the risks of standing on the sidelines have become bigger than the risks (and costs) of getting involved. ESG might now have reached that tipping point. 
	}
\end{quote}
\end{itemize}
\item \textbf{``Definition of Green Finance''} by Nannette Lindenberg, DIE, April 2014

\item \textbf{``Brussels eyes easing bank rules to spur green lending''} in \url{https://luxtimes.lu/european-union/39143-brussels-eyes-easing-bank-rules-to-spur-green-lending}
\begin{itemize}
	\item \textit{But it is likely to stoke a battle with regulators at the European Central Bank, where officials have warned against tampering with rules designed to make bank lending less risky.}
	\item \textit{Valdis Dombrovskis, a vice-president of the commission, told the Financial Times that he wanted to examine a cut to the capital charges imposed on banks' climate-friendly lending. He said the initiative would encourage banks to finance energy-efficient homes, zero-emissions transport and other green investment by reducing the amount of capital they would have to set aside against such lending.}
	\item ECB banking supervision votes against it, green does not mean safe
	\item other measures envisaged by the commission: reform of the emissions trading scheme
	\item \textit{The commission is also at the early stages of discussing a so-called carbon border adjustment that would aim to prevent tighter EU emissions rules from disadvantaging domestic industry compared with overseas competitors, he said. This would entail applying the EU's carbon price to certain categories of imports, starting with a narrow range of industries and then potentially broadening out.}
\end{itemize}

\item \textbf{``Lagarde wants key role for climate change in ECB review''} by martin arnold
28.11.2019
 in Luxembourg times, \url{https://luxtimes.lu/european-union/39156-lagarde-wants-key-role-for-climate-change-in-ecb-review}
\begin{itemize}
	\item \textit{Christine Lagarde is pushing for climate change to be part of a strategic review of the European Central Bank's purpose, spearheading a global drive to make the environment an essential part of monetary policymaking.}
	\item \textit{François Villeroy de Galhau, Banque de France governor and member of the ECB's governing council, said he supported Lagarde's initiative. By increasing energy prices and lowering economic growth, global warming could cause a "stagflationary shock", which meant it was "already part of our price stability mandate," he told the Financial Times.}
	
	\item Without any further delay, the ECB should commit to gradually eliminating carbon-intensive assets from its portfolios, starting with immediate divestment from coal-related assets, reads the letter, whose signatories include Adair Turner, former head of the UK's financial watchdog, and Francesco Papadia, former director-general of market operations at the ECB.
	The ECB's decision to buy bonds of carbon-intensive and fossil fuel-related industries as part of its Euro2.6 trillion asset purchase programme is particularly shocking, the letter says. Climate-impact criteria should be used to screen all assets currently eligible for monetary operations.
	
\end{itemize}
\item \textbf{``The role of green finance in environmental protection: Two aspects of market mechanism and policies''} in Energy Procedia, 2016  by Wang, Zhi
\begin{itemize}
	\item 
\end{itemize}

\item  ESG of Swedish Pension fund: \url{https://www.ap4.se/en/esg/core-values/}
\begin{itemize}
\item 	
\end{itemize}
\item Master's thesis: \textit{Responsible investments in the Swedish pension fund system
A case study of institutional investors}
\item Articles by Ian Gough (mendeley)
\item Nannette Linderberg paper on case studies (printed): Green banking regulation
\item \textbf{``Human values and beliefs and concern about climate change:...''} 2018, Springer by Prati, Pietratoni, Albanesi
\end{itemize}
\subsection{Green Finance and redistribution}
\section{Ethics}
\subsection{Distributional justice}
\subsection{Ethics/distributive justice of climate change}
\section{Combined}
	%------------------------
	% References
	%------------------------
	
	\newpage
	\pagenumbering{Roman}
	%\printbibliography
	\bibliography{bib}
	\bibliographystyle{apa}
	%-----------------------------
	
\end{document}